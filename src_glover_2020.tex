\documentclass[letterpaper,mathserif,english,xcolor=dvipsnames]{beamer}

%\usepackage{beamerthemesplit}

\usetheme{Madrid}
%\usecolortheme{beaver}


\usepackage{times}
\usepackage[T1]{fontenc}
\usepackage[latin1]{inputenc}
\usepackage{amsmath}
\usepackage{color}
\usepackage{graphicx}
\usepackage{amssymb}
\usepackage[all]{xy}
\usepackage{tikz}
\usepackage{mathrsfs} 
%\usepackage{bbold}
%\usepackage{animate}

\usepackage{fp}

\theoremstyle{definition}
%\newtheorem{definition}[definition]{Definition}
\theoremstyle{theorem}
%\newtheorem{fact}[theorem]{Fact}
\newtheorem{question}[theorem]{Question}
\newtheorem{ifact}[theorem]{Important Theorem}
\newtheorem{strategy}[theorem]{Strategy}
\newtheorem{proposition}[theorem]{Proposition}
\newtheorem{idea}[theorem]{Idea}
\newtheorem{conjecture}[theorem]{Conjecture}
\newtheorem{dream}[theorem]{Dream}

%\newtheorem{lemma}[theorem]{Lemma}

\newcommand{\R}{\mathbb{R}}
\newcommand{\N}{\mathbb{N}}
\newcommand{\Z}{\mathbb{Z}}
\newcommand{\C}{\mathbb{C}}
\newcommand{\Q}{\mathbb{Q}}
\newcommand{\G}{\mathbb{G}}
%\newcommand{\L}{\mathbb{L}}
\newcommand{\D}{\mathbb{D}}
\newcommand{\A}{\mathbb{A}}
%\newcommand{\P}{\mathbb{P}}
\newcommand{\B}{\mathbb{B}}
\newcommand{\Hidden}[1]{}
\newcommand{\RS}{\mathcal R_S}

\newcommand{\disp}{\displaystyle}

\newcommand{\defbold}[1]{\textcolor{blue}{\textbf{#1}}}
\newcommand{\bigzero}{\mbox{\normalfont\Large 0}}

\DeclareMathOperator{\pr}{pr}
\DeclareMathOperator{\vol}{vol}

\beamertemplatenavigationsymbolsempty

\title[Non-Backtracking Spectrum of Graphs]{Non-Backtracking Spectrum of Graphs}



\author[Cory Glover]{{\large Cory Glover} \\  \vspace{.2cm}
Brigham Young University \\ Department of Mathematics \\ \vspace{.4cm} {\footnotesize Joint work with\\ Mark Kempton, Brigham Young University}
 } \institute[BYU]


\date[October 25, 2020

]{AMS Western Sectional 2020\\Provo, UT\\October 25, 2020
}


% Define style for nodes
%\tikzstyle{every node}=[circle, fill=black!50,
%                        inner sep=0pt, minimum width=2pt]
                   
%\tikzstyle{every node}=[circle, draw, fill=black!50,
 %                       inner sep=0pt, minimum width=2pt]
\tikzstyle{every node}=[circle,fill=black,inner sep=1pt]


\newcommand{\F}{\mathbb{F}}


\begin{document}

\frame{\titlepage}



%\section[Outline]{}
%\frame{\tableofcontents}
\colorlet{nodecolor}{red}

%\tikzstyle{every node}=[fill=white,
%                        inner sep=0pt, minimum width=4pt]


%\section{Background}
%\subsection{Original Question}
%\subsection{New Question}

\frame{\frametitle{Non-backtracking Random Walks}

\begin{definition}[Random Walk]
Let $G=(V,E)$ be a graph. A \emph{random walk} across $G$ is a walk across a graph where $v_{i+1}$ is chosen uniformly at random from the set of neighbors of $v_i$.\end{definition}
\pause
%}
%
%\frame{\frametitle{Non-backtracking Random Walks}
%
%\begin{definition}[Random Walk]
%Let $G=(V,E)$ be a graph with $n$ vertices and $m$ edges. A \emph{random walk} across $G$ is a walk across a graph where $v_{i+1}$ is chosen uniformly at random from the set of neighbors of $v_i$.\end{definition}

\begin{definition}[Non-backtracking Random Walk]
Let $G=(V,E)$ be a graph. A \emph{non-backtracking random walk} across $G$ is a walk across a graph where $v_{i+1}$ is chosen uniformly at random from the set of neighbors of $v_i$ excluding $v_{i-1}$.\end{definition}

}

\frame{\frametitle{Non-backtracking Matrix}

\begin{definition}[NB Matrix]
Let $B\in M_{2m}$ be the non-backtracking matrix.
Then
\[B((u,v),(x,y))=\begin{cases}1&v=x\text{ and }u\neq y\\0&\text{otherwise}\end{cases}.\]
\end{definition}

\centering
\includegraphics{nb_cycle.pdf}

}

\frame{\frametitle{Spectrum of $B$}

\textbf{Question:}  What do we know about the spectrum of $B$, $\sigma(B)$, for a given graph $G$?


}

\frame{\frametitle{Spectrum of $B$}

\textbf{Question:}  What do we know about the spectrum of $B$, $\sigma(B)$, for a given graph $G$?

\begin{theorem}[Ihara's Theorem]
Let $G$ be a graph with adjacency matrix $A$, degree matrix $D$, and non-backtracking matrix $B$.
Then 
\[\text{det}(\mu I-B)=(1-\mu^2)^{m-n}\text{det}(\mu^2I-\mu A+(D-I)).\]
\end{theorem}

}

\frame{\frametitle{$d$-regular Graphs}


Let $G$ be a $d$-regular graph with adjacency matrix $A$ and non-backtracking matrix $B$.
Then 
\[\pm 1,\frac{\lambda_i\pm\sqrt{\lambda_i^2-4(d-1)}}{2}\]
are the eigenvalues of $B$ where $\lambda_i\in\sigma(A)$ and $\pm 1$ each have multiplicity $m-n$.


}


\frame{\frametitle{The Matrix $K$}

From Krzakala et. al. (2013), the matrix $K$ of a graph $G$
\[K=\begin{pmatrix}A&D-I\\-I&\mathbf{0}\end{pmatrix}\]
has characteristic polynomial
\[\text{det}(\mu I-K)=\text{det}(\mu^2I-\mu A+(D-I)).\]

}

\frame{\frametitle{Decomposition of $B$}
\begin{center}
\begin{align*}
S((u,v),x)&=\begin{cases}1&v=x\\0&\text{otherwise}\end{cases}&T(x,(u,v))&=\begin{cases}1&u=x\\0&\text{otherwise}\end{cases}
\end{align*}

\[\tau((i,j),(k,l))=\begin{cases}1&i=l\text{ and }j=k\\0&\text{otherwise}\end{cases}\]
\end{center}

\pause

\begin{center}
\begin{align*}
B&=ST-\tau&D&=T\tau S&A&=TS
\end{align*}
\end{center}

\pause

\begin{center}
\begin{align*}
&&B\begin{bmatrix}S&T^T\end{bmatrix}&=\begin{bmatrix}S&T^T\end{bmatrix}K&&
\end{align*}
\end{center}
}

\frame{\frametitle{Decomposition of $B$}
\begin{theorem}[Lubetzky and Peres, 2010]
Let $G$ be a connected $d$-regular graph ($d\geq 3$) on $n$ vertices.
Let $N=dn$ and let $\lambda_i\in\sigma(A)$, with $\lambda_1=d$. Then the operator $B$ is unitarily similar to
\[\Lambda=\text{diag}\Biggl(d-1,\begin{bmatrix}\theta_2&\alpha_2\\0&\theta_2'\end{bmatrix},...,\begin{bmatrix}\theta_n&\alpha_n\\0&\theta_n'\end{bmatrix},-1,...-1,1,...,1\Biggr)\]
where $|\alpha_i|<2(d-1)$ for all $i$, $\theta_i$ and $\theta_i'$ are defined as the solutions of
\[\theta^2-\lambda_i\theta+d-1=0\]
and $-1$ has multiplicity $N/2-n$ and $1$ has multiplicity $N/2-n+1$.
\end{theorem}

}


\frame{\frametitle{Decomposition of $B$}
\begin{theorem}
Let $G$ be a connected graph and $B$ its non-backtracking matrix.  Define $\mathscr{E}_i$ to be the eigenspace of $i$ for $\tau$ ($i=\pm$). Let $R\in M_{2m\times 2(m-n)}$ where the columns of $R$ are linearly independent and the first $m-n$ columns are taken from $\mathscr{E}_{-1}\cap\text{Null}(ST)$ and the second $m-n$ columns are taken from $\mathscr{E}_1\cap\text{Null}(ST)$.
Then
\[BX=X\begin{bmatrix}K&0&0\\0&I_{m-n}&0\\0&0&-I_{m-n}\end{bmatrix}\]
where $X=\begin{bmatrix}S&T^T&R\end{bmatrix}$.
\end{theorem}
}

\frame{\frametitle{The Matrix $K$}
\begin{proposition}
Let $G$ be a graph and $K$ as defined previously. Then the following are true:
\begin{enumerate}[(i)]
\item Every eigenvalue-eigenvector of $K$ is of the form $(\mu, \begin{bmatrix}-\mu y&y\end{bmatrix}^T)$
\item $1\in\sigma(K)$ with algebraic multiplicity equal to the number of connected components of $G$,
\item the nullity of $K$ is the number of degree one vertices, and
\item $K^{-1}=\begin{bmatrix}0&-I\\(D-I)^{-1}&(D-I)^{-1}A\end{bmatrix}$ when $d\geq 2$.
\end{enumerate}
\end{proposition}
}

\frame{\frametitle{Eigenvalues of $K$}

Let $\mu\in\sigma(K)$. Then
\begin{align*}
\mu\begin{bmatrix}-\mu y\\y\end{bmatrix}&=\begin{bmatrix}A&D-I\\-I&0\end{bmatrix}\begin{bmatrix}-\mu y\\y\end{bmatrix}\\
0&=\mu^2y-\mu Ay+(D-I)y 
\end{align*}

\pause 

Let $\lambda\in\sigma(A)$ with eigenvector $x$. If $x^Ty\neq 0$, scale $x$ such that $x^Ty=1$. Then
\begin{align*}
0&=\mu^2-\mu\lambda+x^T(D-I)y\\
\mu&=\frac{\lambda\pm\sqrt{\lambda^2-4x^T(D-I)y}}{2}.
\end{align*}

}

\frame{\frametitle{Spectral Radius of $K$}
\begin{proposition}
Let $G$ be a connected graph that is not a cycle and $d\geq 2$. Then $B$ is irreducible.
\end{proposition}

\begin{lemma}
Let $G$ be a connected graph that is not a cycle and $d\geq 2$. Then $\rho(K)>1$ and there is a positive vector $y$ such that $K\begin{bmatrix}-\rho(K)y&y\end{bmatrix}^T=\rho(K)\begin{bmatrix}-\rho(K)y&y\end{bmatrix}^T$.
\end{lemma}

}

\frame{\frametitle{Spectral Radius of $K$}

\begin{lemma}
Let $G$ be a connected graph that is not a cycle and $d\geq 2$. Then $\rho(K)>1$ and there is a positive vector $y$ such that $K\begin{bmatrix}-\rho(K)y&y\end{bmatrix}^T=\rho(K)\begin{bmatrix}-\rho(K)y&y\end{bmatrix}^T$.
\end{lemma}

Sketch:

\begin{itemize}
\item Break into regular and non-regular case for $\rho(K)>1$
\end{itemize}

}

\frame{\frametitle{Spectral Radius of $K$}

\begin{lemma}
Let $G$ be a connected graph that is not a cycle and $d\geq 2$. Then $\rho(K)>1$ and there is a positive vector $y$ such that $K\begin{bmatrix}-\rho(K)y&y\end{bmatrix}^T=\rho(K)\begin{bmatrix}-\rho(K)y&y\end{bmatrix}^T$.
\end{lemma}

Sketch:

\begin{itemize}
\item Break into regular and non-regular case for $\rho(K)>1$
\begin{itemize}
\item For regular case use fact that $d\geq 3$
\item For non-regular case use minimum row sums
\end{itemize}
\end{itemize}

}

\frame{\frametitle{Spectral Radius of $K$}

\begin{lemma}
Let $G$ be a connected graph that is not a cycle and $d\geq 2$. Then $\rho(K)>1$ and there is a positive vector $y$ such that $K\begin{bmatrix}-\rho(K)y&y\end{bmatrix}^T=\rho(K)\begin{bmatrix}-\rho(K)y&y\end{bmatrix}^T$.
\end{lemma}

Sketch:

\begin{itemize}
\item Break into regular and non-regular case for $\rho(K)>1$
\begin{itemize}
\item For regular case use fact that $d\geq 3$
\item For non-regular case use minimum row sums
\end{itemize}
\item By Perron-Frobenius and since $B\begin{bmatrix}S&T^T\end{bmatrix}=\begin{bmatrix}S&T^T\end{bmatrix}K$, we know that $T^Ty\succ \rho(K)Sy$ or $\rho(K)Sy\succ T^Ty$
\end{itemize}

}

\frame{\frametitle{Spectral Radius of $K$}

\begin{lemma}
Let $G$ be a connected graph that is not a cycle and $d\geq 2$. Then $\rho(K)>1$ and there is a positive vector $y$ such that $K\begin{bmatrix}-\rho(K)y&y\end{bmatrix}^T=\rho(K)\begin{bmatrix}-\rho(K)y&y\end{bmatrix}^T$.
\end{lemma}

Sketch:

\begin{itemize}
\item Break into regular and non-regular case for $\rho(K)>1$
\begin{itemize}
\item For regular case use fact that $d\geq 3$
\item For non-regular case use minimum row sums
\end{itemize}
\item By Perron-Frobenius and since $B\begin{bmatrix}S&T^T\end{bmatrix}=\begin{bmatrix}S&T^T\end{bmatrix}K$, we know that $T^Ty\succ \rho(K)Sy$ or $\rho(K)Sy\succ T^Ty$
\item BWOC, show $\rho(K)Sy\succ T^Ty$.
\end{itemize}

}

\frame{\frametitle{Spectral Radius of $K$}

\begin{lemma}
Let $G$ be a connected graph that is not a cycle and $d\geq 2$. Then $\rho(K)>1$ and there is a positive vector $y$ such that $K\begin{bmatrix}-\rho(K)y&y\end{bmatrix}^T=\rho(K)\begin{bmatrix}-\rho(K)y&y\end{bmatrix}^T$.
\end{lemma}

Sketch:

\begin{itemize}
\item Break into regular and non-regular case for $\rho(K)>1$
\begin{itemize}
\item For regular case use fact that $d\geq 3$
\item For non-regular case use minimum row sums
\end{itemize}
\item By Perron-Frobenius and since $B\begin{bmatrix}S&T^T\end{bmatrix}=\begin{bmatrix}S&T^T\end{bmatrix}K$, we know that $T^Ty\succ \rho(K)Sy$ or $\rho(K)Sy\succ T^Ty$
\item BWOC, show $\rho(K)Sy\succ T^Ty$.
\item Use the fact that $S$ and $T^T$ have one nonzero entry in each row.
\end{itemize}

}

\frame{\frametitle{Spectral Radius of $K$}
\begin{theorem}
Let $G$ be a connected graph, $A$ its adjacency matrix, $D$ the degree matrix, and $B$ the non-backtracking matrix.
If $\rho(A)\geq 2x^T(D-I)y$, then 
\[\rho(B)\leq\frac{\rho(A)+\sqrt{\rho(A)^2-4(d_{min}-1)}}{2}.\]
\end{theorem}

\pause

Sketch:

\begin{itemize}
\item If $Ax=\rho(A)x$ and $K\begin{bmatrix}-\rho(K)y&y\end{bmatrix}^T=\rho(K)\begin{bmatrix}-\rho(K)y&y\end{bmatrix}^T$, we know $x^Ty\neq 0$ from previous lemma
\item Apply $\mu=\frac{\lambda\pm\sqrt{\lambda^2-4x^T(D-I)y}}{2}$
\end{itemize}

}

\frame{\frametitle{Spectral Radius of $K$}
From a bound by Das and Kumar (2004), we get
\begin{corollary}
Let $G$ be a connected graph and $B$ its non-backtracking matrix. If $\rho(A)\geq 2\sqrt{x^T(D-I)y}$,
\[\rho(B)\leq\frac{\sqrt{2m-n-1}+\sqrt{2m-n-4d_{min}+1}}{2}.\]

\end{corollary}

}

\frame{\frametitle{Smallest Eigenvalue of $K$}
\begin{proposition}
Let $G$ be a connected graph with $d\geq 2$.
Then $|\mu|\geq 1$ for all $\mu\in\sigma(K)$.
\end{proposition}
\pause
Sketch:

\begin{itemize}
\item Build equation $\mu=\frac{y^TAy\pm\sqrt{(y^TAy)^2-4y^T(D-I)y}}{2}$
\end{itemize}

}

\frame{\frametitle{Smallest Eigenvalue of $K$}
\begin{proposition}
Let $G$ be a connected graph with $d\geq 2$.
Then $|\mu|\geq 1$ for all $\mu\in\sigma(K)$.
\end{proposition}

Sketch:

\begin{itemize}
\item Build equation $\mu=\frac{y^TAy\pm\sqrt{(y^TAy)^2-4y^T(D-I)y}}{2}$
\item When $(y^TAy)^2\leq 4y^T(D-I)y$, use $d\geq 2$
\end{itemize}

}
\frame{\frametitle{Smallest Eigenvalue of $K$}
\begin{proposition}
Let $G$ be a connected graph with $d_{\min}\geq 2$.
Then $|\mu|\geq 1$ for all $\mu\in\sigma(K)$.
\end{proposition}

Sketch:

\begin{itemize}
\item Build equation $\mu=\frac{y^TAy\pm\sqrt{(y^TAy)^2-4y^T(D-I)y}}{2}$
\item When $(y^TAy)^2\leq 4y^T(D-I)y$, use $d_{\min}\geq 2$
\item When $(y^TAy)^2>4y^T(D-I)y$, use $d_{\min}\geq 2$ and the positive semi-definiteness of the Laplacian
\end{itemize}

}

\frame{\frametitle{Bipartite Graphs}
\begin{theorem}
Let $G$ be a connected graph and $B$ its non-backtracking matrix. The following are equivalent:
\begin{enumerate}
\item $G$ is a bipartite graph,
\item $\sigma(K)$ is symmetric,
\item $\sigma(B)$ is symmetric,
\item $-1\in\sigma(K)$,
\item $\lambda_n=-\lambda_1$ for $\lambda_i\in\sigma(K)$, and
\item $\mu_n=-\mu_1$ for $\mu_i\in\sigma(B)$.
\end{enumerate}
\end{theorem}


}

\frame{\frametitle{Conclusion}
What we know:
\begin{itemize}
\item We can use $K$ to understand the spectrum of $B$
\item $B$ can be related to a block diagonal matrix showing more explicitly the spectrum of $B$
\item We can use $K$ to bound the spectral radius of $B$
\item The spectrum of $B$ and $K$ can indicate whether $G$ is bipartite
\end{itemize}


}

\frame{\frametitle{Conclusion}

Future Work:
\begin{itemize}
\item Can we identify the spectral gap of $B$?
\item Can we relate the transition probability matrix of $G$ to the non-backtracking transition probability matrix?
\end{itemize}
}

\frame{\frametitle{The End}

\begin{center}

{\Huge Thank You!}

\end{center}
}


\end{document}