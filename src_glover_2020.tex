\documentclass[letterpaper,mathserif,english,xcolor=dvipsnames]{beamer}

%\usepackage{beamerthemesplit}

\usetheme{Madrid}
%\usecolortheme{beaver}


\usepackage{times}
\usepackage[T1]{fontenc}
\usepackage[latin1]{inputenc}
\usepackage{amsmath}
\usepackage{color}
\usepackage{graphicx}
\usepackage{amssymb}
\usepackage[all]{xy}
\usepackage{tikz}
%\usepackage{bbold}
%\usepackage{animate}

\usepackage{fp}

\theoremstyle{definition}
%\newtheorem{definition}[definition]{Definition}
\theoremstyle{theorem}
%\newtheorem{fact}[theorem]{Fact}
\newtheorem{question}[theorem]{Question}
\newtheorem{ifact}[theorem]{Important Theorem}
\newtheorem{strategy}[theorem]{Strategy}
\newtheorem{proposition}[theorem]{Proposition}
\newtheorem{idea}[theorem]{Idea}
\newtheorem{conjecture}[theorem]{Conjecture}
\newtheorem{dream}[theorem]{Dream}

%\newtheorem{lemma}[theorem]{Lemma}

\newcommand{\R}{\mathbb{R}}
\newcommand{\N}{\mathbb{N}}
\newcommand{\Z}{\mathbb{Z}}
\newcommand{\C}{\mathbb{C}}
\newcommand{\Q}{\mathbb{Q}}
\newcommand{\G}{\mathbb{G}}
%\newcommand{\L}{\mathbb{L}}
\newcommand{\D}{\mathbb{D}}
\newcommand{\A}{\mathbb{A}}
%\newcommand{\P}{\mathbb{P}}
\newcommand{\B}{\mathbb{B}}
\newcommand{\Hidden}[1]{}
\newcommand{\RS}{\mathcal R_S}

\newcommand{\disp}{\displaystyle}

\newcommand{\defbold}[1]{\textcolor{blue}{\textbf{#1}}}
\newcommand{\bigzero}{\mbox{\normalfont\Large 0}}

\DeclareMathOperator{\pr}{pr}
\DeclareMathOperator{\vol}{vol}

\beamertemplatenavigationsymbolsempty

\title[Non-Backtraking Random Walks]{Non-backtracking Random Walks}



\author[Cory Glover]{{\large Cory Glover} \\  \vspace{.2cm}
Brigham Young University \\ Department of Mathematics \\ \vspace{.4cm} {\footnotesize Joint work with\\ Mark Kempton, Brigham Young University\\Tyler Jones, Brigham Young University}
 } \institute[BYU]


\date[February 29, 2020

]{SCR 2020\\Provo, UT\\February 29,2020
}


% Define style for nodes
%\tikzstyle{every node}=[circle, fill=black!50,
%                        inner sep=0pt, minimum width=2pt]
                   
%\tikzstyle{every node}=[circle, draw, fill=black!50,
 %                       inner sep=0pt, minimum width=2pt]
\tikzstyle{every node}=[circle,fill=black,inner sep=1pt]


\newcommand{\F}{\mathbb{F}}


\begin{document}

\frame{\titlepage}



%\section[Outline]{}
%\frame{\tableofcontents}
\colorlet{nodecolor}{red}

%\tikzstyle{every node}=[fill=white,
%                        inner sep=0pt, minimum width=4pt]


%\section{Background}
%\subsection{Original Question}
%\subsection{New Question}

\frame{\frametitle{Problem}

\begin{center}
Is the mixing rate of a non-backtracking random walk faster than that of a simple random walk on a graph?
\end{center}

}

\frame{\frametitle{Mixing Rate}
\begin{definition}
Let $P$ be the transition probability matrix of a graph $G$ and let $\pi$ be the stationary distribution of a random walk on $G$. Then the \textbf{\textcolor{blue}{mixing rate}} of a random walk on a graph $G$ is defined pointwise by
\[\rho=\limsup_{t\rightarrow\infty}\max_{u,v}|P^t(u,v)-\pi(v)|^{1/t}.\]
\end{definition}
}

\frame{\frametitle{Mixing Rate and Eigenvalues}

\begin{theorem}[Lovasz 1993]
Let $G$ be a connected non-bipartite graph with transition probability matrix $P$. Let $\mu_1\geq\mu_2\geq\dotsb\geq\mu_n$ be the eigenvalues of $P$. Then the mixing rate of $G$ is $\rho=\max\{|\mu_2|,|\mu_n|\}$.
\end{theorem}


}


\frame{\frametitle{Hashimoto Matrix}



\begin{definition}
The \textbf{\textcolor{blue}{Hashimoto Matrix}} of a graph $G$ is 
\begin{align}
B((u,v),(x,y))&=\begin{cases}1&v=x\text{ and }u\neq y\\0&\text{otherwise}\end{cases}
\end{align}
\end{definition}

}

\frame{\frametitle{Ihara's Theorem}

\begin{theorem}[Ihara]
Given a graph $G$ with $n$ vertices and $m$ edges, define $B$ to be the Hashimoto matrix of $G$. Let $A$ be the adjacency matrix of $G$ and let $D$ be the diagonal degree matrix. Then
\[\text{det}(I-\mu B)=(1-u^2)^{m-n}\text{det}(I-\mu A+\mu^2(D-I)).\]
\end{theorem}
\smallskip
Thus, the eigenvalues of $B$ are $\mu$.


}


\frame{\frametitle{The Matrix $K$}

From Krzakala et. al. (2013), the matrix $K$ of a graph $G$
\[K=\begin{pmatrix}A&D-I\\-I&\mathbf{0}\end{pmatrix}\]
is a invariant subspace of $B$.

}

\frame{\frametitle{The Matrix $K$}

Define the following matrices: \[S((u,v),x)=\begin{cases}1&v=x\\0&\text{otherwise}\end{cases}\] 
 \[T(x,(u,v))=\begin{cases}1&u=x\\0&\text{otherwise}\end{cases}.\]
 \bigskip
Then
\[B\begin{pmatrix}S&T^T\end{pmatrix}v=\begin{pmatrix}S&T^T\end{pmatrix}Kv.\]
\smallskip
If $v$ is an eigenvector of $K$ with eigenvalue $\mu$, then
\[B\begin{pmatrix}S&T^T\end{pmatrix}v=\mu\begin{pmatrix}S&T^T\end{pmatrix}v.\]

}




\frame{\frametitle{Eigenvalues of $K$}
If $\begin{pmatrix}x&y\end{pmatrix}^T$ is an eigenvector of $K$,
\begin{align*}
\begin{pmatrix}A&D-I\\-I&\mathbf{0}\end{pmatrix}\begin{pmatrix}x\\y\end{pmatrix}&=\mu\begin{pmatrix}x\\y\end{pmatrix},
\end{align*}
then $x=-\mu y$.
\smallskip
This means that $\mu^2y-\mu Ay+(D-I)y=\mathbf{0}$ for all $\mu\in\sigma(K)$.
}

\frame{\frametitle{Ihara's Theorem}



\begin{theorem}[Ihara, 1966]
Given a graph $G$ and the associated Hasimoto matrix $B$, adjacency matrix $A$, and diagonal degree matrix $D$, the following is true:
\[\text{det}(I-\mu B)=(1-u^2)^{m-n}\text{det}(I-u A-u^2(D-I)).\]
\end{theorem}
\bigskip
So the eigenvalues of $B$ are the $\frac{1}{u}$. 
}

\frame{\frametitle{Relationship Between $A$ and $K$}

Let $\mathbf{x}$ be the eigenvector associated with $\lambda_2$ (the second largest eigenvalue) of $A$. Let $\mu_2$ be the second largest eigenvalue of $K$ such that $K\begin{pmatrix}-\mu_2\mathbf{y}&\mathbf{y}\end{pmatrix}^T=\mu_2\begin{pmatrix}-\mu_2\mathbf{y}&\mathbf{y}\end{pmatrix}^T$. Scale $\mathbf{x}$ such that $\mathbf{x}^T\mathbf{y}=1$. 

\bigskip

Then,
\begin{align*}
\mu_2^2\mathbf{x}^T\mathbf{y}-\mu_2\mathbf{x}^TA\mathbf{y}+\mathbf{x}^T(D-I)\mathbf{y}&=0,\\
\mu_2^2-\mu_2\lambda_2+\mathbf{x}^T(D-I)\mathbf{y}&=0.
\end{align*}

\bigskip

Solving for $\mu_2$,
\[\mu_2=\frac{\lambda_2\pm\sqrt{\lambda_2^2-4\mathbf{x}^T(D-I)\mathbf{y}}}{2}.\]
}

\frame{ \frametitle{Questions That Arise}

\[\mu=\frac{\lambda\pm\sqrt{\lambda^2-4\mathbf{x}^T(D-I)\mathbf{y}}}{2}\]

\bigskip

\begin{itemize}
\item How are $\mu$ and $\lambda$ related?
\item Is it $+$ or $-$?
\item Can we get all eigenvalues $\mu$?
\item How to convert $\mu$ to $\rho$?
\end{itemize}

  
}

\frame{ \frametitle{Relating $\mu$ and $\lambda$}

Assume that $\lambda \geq 2\sqrt{\mathbf{x}^T(D-I)\mathbf{y}}$:
\begin{align*}
\mu&=\frac{\lambda+\sqrt{\lambda^2-4\mathbf{x}^T(D-I)\mathbf{y}}}{2}\leq\frac{\lambda+\lambda}{2}=\lambda\\
\mu&=\frac{\lambda-\sqrt{\lambda^2-4\mathbf{x}^T(D-I)\mathbf{y}}}{2}\leq\frac{\lambda}{2}.
\end{align*}
In either case $\mu\leq\lambda$.


}



\frame{ \frametitle{Bipartite Graphs and $K$}

INSERT PICTURE OF BIPARTITE GRAPH HERE

}

\frame{\frametitle{Bipartite Graphs and $K$}

\begin{theorem}
The following are equivalent for a graph $G$:
\begin{enumerate}
\item $G$ is bipartite,
\item The spectrum of $B$ is symmetric,
\item The spectrum of $K$ is symmetric,
\item $-1$ is an eigenvalue of $K$.
\end{enumerate}
\end{theorem}

}





\frame{\frametitle{Bipartite $\Rightarrow$ $\sigma(B)$ Is Symmetric}
If $G$ is bipartite, then 
\[B=\begin{pmatrix}0&B_2\\B_1&0\end{pmatrix}.\]
Let $\lambda\in\sigma(B)$. Then,
\begin{align}
B\begin{pmatrix}x&y\end{pmatrix}^T&=\lambda\begin{pmatrix}x&y\end{pmatrix}^T\\
\begin{pmatrix}B_2y&B_1x\end{pmatrix}^T&=\lambda\begin{pmatrix}x&y\end{pmatrix}^T.
\end{align}
This means that
\begin{align}
B\begin{pmatrix}x&-y\end{pmatrix}^T&=\begin{pmatrix}-B_2y&B_1x\end{pmatrix}^T\\
&=\begin{pmatrix}-\lambda x&\lambda y\end{pmatrix}^T\\
&=-\lambda\begin{pmatrix}x&-y\end{pmatrix}^T.
\end{align}
So $-\lambda\in\sigma(B)$.
}

\frame{\frametitle{$\sigma(B)$ Is Symmetric $\Rightarrow$ $\sigma(K)$ Is Symmetric}
\begin{theorem}[Ihara, 1966]
Given a graph $G$ and the associated Hasimoto matrix $B$, adjacency matrix $A$, and diagonal degree matrix $D$, the following is true:
\[\text{det}(I-\mu B)=(1-u^2)^{m-n}\text{det}(I-u A-u^2(D-I)).\]
\end{theorem}

}

\frame{ \frametitle{$\sigma(K)$ is symmetric $\Rightarrow$ $\sigma(B)$ is symmetric.}

\begin{theorem}[Ihara, 1966]
Given a graph $G$ and the associated Hasimoto matrix $B$, adjacency matrix $A$, and diagonal degree matrix $D$, the following is true:
\[\text{det}(I-\mu B)=(1-u^2)^{m-n}\text{det}(I-u A-u^2(D-I)).\]
\end{theorem}

}


\frame{ \frametitle{$\sigma(B)$ is symmetric $\Rightarrow$ $G$ is bipartite}
Consider $B$ as the adjacency matrix of its own graph.



}

\frame{\frametitle{Paths: Cospectrality and Fractional Cospectrality}
When can two vertices of a path be cospectral or fractionally cospectral?

\pause

\begin{lemma}
Let $u,v$ be vertices of a path on $n$ vertices $P_n$
\begin{itemize}
\item $u$ and $v$ are cospectral if and only if $\{u,v\}=\{i,n-i\}$.\pause
 \item $u$ and $v$ are fractionally cospectral (and not cospectral) if and only if $n=5d-1$ for some positive integer $d$, and $\{u,v\}=\{d,3d\}$ or (by symmetry) $\{u,v\}=\{2d,4d\}$.
\end{itemize}
\end{lemma}

\begin{center}
\begin{tikzpicture}
\draw (-4,0)node{}--(-3,0)node{}--(-2,0)node{}--(-1,0)node{}--(0,0)node{}--(1,0)node{}--(2,0)node{}--(3,0)node{}--(4,0)node{};
\draw (-3,0)node[fill=none,below]{$u$} (1,0)node[fill=none,below]{$v$};
\end{tikzpicture}
\end{center}

}

\frame{\frametitle{Fractional Cospectrality in Paths}

We prove this by counting walks in paths

\begin{align*}
A^{2d+2j}(u,u) &= \binom{2d+2j}{d+j}-\binom{2d+2j}{j}\\
A^{2d+2j}(v,v) &= \binom{2d+2j}{d+j}\\
A^{2d+2j}(u,v) &= \binom{2d+2j}{j}.
\end{align*}


}

\frame{\frametitle{Paths: Eigenvalue Condition}

The eigenvalues of $P_{5d-1}$ are
\[
\lambda_j=2\cos\frac{\pi j}{5d},~~~~j=1,...,5d-1
\]

\pause

\begin{lemma}
Let $m$ be an odd integer, $0\leq a< k$ integers.  Then
\[
\sum_{i=0}^{m-1}(-1)^i\cos\left(\frac{(a+ik)\pi}{km}\right)=0.
\]
\end{lemma}

\pause

Eigenvalue condition satisfied if and only if $d=2^k$ for some $k$.

}

\frame{\frametitle{PGFR in Paths}

\begin{theorem}
Vertices $u$ and $v$ of $P_n$ exhibit PGFR if and only if we are in one of the following cases:
\begin{itemize}
\item There is PGST from $u$ to $v$ (characterized previously).
\item $n=5\cdot 2^k-1$ and $\{u,v\} = \{2^k,3\cdot2^k\}$ or $ \{2\cdot2^k,4\cdot2^k\}$.
\end{itemize}
\end{theorem}

}

\frame{\frametitle{Future Work}

\begin{itemize}
\item What other graphs have fractionally cospectral pairs of nodes?  Which of these exhibit FR/PGFR?\pause

\item Weighted paths?\pause

\item FR or PGFR among more than 2 nodes in paths?
\end{itemize}

}

\frame{\frametitle{The End}

\begin{center}

{\Huge Thank You!}

\end{center}
}


\end{document}